\documentclass{article}
\usepackage{amsmath, tikz, tcolorbox, array, multicol, sfmath, enumerate, pgfplots}
\renewcommand{\familydefault}{\sfdefault}
\pgfplotsset{compat=newest}
\usetikzlibrary{arrows.meta}
\everymath{\displaystyle}
\tikzset{>=stealth}
\usepackage[top = 0.25in, bottom = 0.25in, left = 1in, right = 1in]{geometry}
\pagestyle{empty}
\raggedright

\newcounter{example}[section]
\newenvironment{example}[1][]{\refstepcounter{example}\par\medskip
   {\color{red}\textbf{Example~\theexample. #1}}}{\medskip}

\begin{document}

\section*{Radical Equations}

\begin{tcolorbox}[colframe=orange!70!white, coltitle=black, title=\textbf{Summary}]
\begin{enumerate}
    \item When solving radical equations, get the radical by itself and raise both sides to the root's power.
    \item You will need to check that your answers don't result in square roots of negative values.
\end{enumerate}
\end{tcolorbox}
\vspace{1in}

\begin{example}
Solve each. Check your answers for extraneous solutions.
\begin{enumerate}[(a)]
    \item $\sqrt{2x-3}=9$   \vfill 
    \item $\sqrt{-10x-1}+3x=0$  \vfill \newpage 
    \item $\sqrt[3]{x+1}+5=3$   \vfill 
    \item $\sqrt{4-x}=x-2$  \vfill 
\end{enumerate}
\end{example}

\end{document}
