\documentclass{article}
\usepackage{amsmath, amssymb, bm, tikz, mathtools, tcolorbox, array, sfmath, enumerate, multicol, pgfplots}
\renewcommand{\familydefault}{\sfdefault}
\pgfplotsset{compat=newest}
\usetikzlibrary{arrows.meta}
\everymath{\displaystyle}
\tikzset{>=stealth}
\tikzstyle{input} = [circle, text centered, radius = 1cm, draw = black]
\tikzstyle{function} = [rectangle, text centered, minimum width = 2cm, minimum height = 1cm, draw = black]
\usepackage[top = 0.25in, bottom = 0.25in, left = 1in, right = 1in]{geometry}
\pagestyle{empty}
\raggedright

\newcounter{example}[section]
\newenvironment{example}[1][]{\refstepcounter{example}\par\medskip
   {\color{red}\textbf{Example~\theexample. #1}}}{\medskip}

\begin{document}

\section*{Simplifying Rational Expressions}

\begin{tcolorbox}[colframe=orange!70!white, coltitle=black, title=\textbf{Summary}]
\begin{enumerate}
    \item Simplify rational expressions by factoring the numerator and denominator; then divide common factors.
\end{enumerate}
\end{tcolorbox}
\bigskip

\textsc{To simplify rational expressions:}
\begin{enumerate}
    \item Factor the numerator and denominator completely.
    \item Divide any common factors.
\end{enumerate}
\bigskip 

\begin{example}
Simplify each completely.
\begin{enumerate}[(a)]
\begin{multicols}{2}
    \item $\frac{x+1}{x^2+4x+3}$
    \item $\frac{x^2+7x+10}{5x+10}$
\end{multicols}
\vfill 
\begin{multicols}{2}
    \item $\frac{x^2-7x-18}{2x^2+3x-2}$
    \item $\frac{x^2-2x-15}{3x^2+8x-3}$ 
\end{multicols}
\end{enumerate}
\end{example}

\vfill 

\end{document}
