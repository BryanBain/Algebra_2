\documentclass{article}
\usepackage{amsmath, bm, enumerate, multicol, pgfplots, sfmath, tcolorbox}
\pgfplotsset{compat = newest}
\renewcommand{\familydefault}{\sfdefault}
\usepackage[top=0.5in, right=1in, bottom=0.5in, left=1in]{geometry}
\pagestyle{empty}
\raggedright
\everymath{\displaystyle}

\newcounter{example}[section]
\newenvironment{example}[1][]{\refstepcounter{example}\par\medskip
   {\color{red}\textbf{Example~\theexample. #1}}}{\medskip}

\begin{document}

\section*{Rational Exponents}

\begin{tcolorbox}[colframe=orange!70!white, colback = white, coltitle=black, title=\textbf{Summary}]
\begin{enumerate}
    \item $\sqrt[root]{x^{power}} = x^{power/root}$
\end{enumerate}
\end{tcolorbox}
\bigskip 

\begin{tcolorbox}[colframe=green!60!black, colback = white,title=\textbf{Radicand}]
For $\sqrt[n]{a}$, the \textbf{radicand} is $\bm{a}$.
\end{tcolorbox}
\bigskip

We are going to build the intuition of rational exponents.

\begin{itemize}
\begin{multicols}{3}
    \item $(12^5)^9 = $
    \item $(12^6)^7 = $
    \item $(12^{\tfrac{1}{7}})^7 = $
\end{multicols}
\vspace{0.25in}
    \item Let $x = 12^{\tfrac{1}{7}}$. Substitute $x$ into the equation below
    \[
    (12^{\tfrac{1}{7}})^7 = 12
    \]
\vspace{0.25in} 
    \item Solve for $x$ using radicals.
    \[
    x^7 = 12
    \]
\vspace{0.25in}
    \item We have $x = 12^{\tfrac{1}{7}}$ and $x = \sqrt[7]{12}$:
    \[
    \boxed{12^{\tfrac{1}{7}} = \sqrt[7]{12}}
    \]
\end{itemize}

% What is $\left(\sqrt[3]{64}\right)^2$? \\[0.5in]

% Step 1: Evaluate $\sqrt[3]{64} = 4$ \newline\\
% Step 2: Raise 4 to the 2nd power: $4^2 = 16$.
% \vfill 

% To put things visually, suppose the block below represents 64:  \newline\\

% \begin{center}
%     \begin{tikzpicture}[scale=0.5]
%     \draw (0,0) rectangle (15,1);
%     \node at (0,0) [below] {1};
%     \node at (15,0) [below] {64};
%     \end{tikzpicture}
% \end{center}
% \vfill 

% Since we are taking the cube root, we can divide the large block up into 3 equal blocks where {\color{blue}\textbf{we multiply by 4 to get each new number}}. 
% \begin{center}
%     \begin{tikzpicture}[scale=0.5]
%     \draw (0,0) rectangle (15,1);
%     \node at (0,0) [anchor=north] {1};
%     \node at (15,0) [anchor=north] {64};
%     \draw (5,0) -- (5,1);
%     \node at (5,0) [anchor=north] {4};
%     \draw (10,0) -- (10,1);
%     \node at (10,0) [anchor=north] {16};
%     \end{tikzpicture}
% \end{center}
% \vspace{0.5in}

% \newpage 

% The exponent of 2 in $\left(\sqrt[3]{64}\right)^2$ means that we shade 2 of the blocks:    \\[0.5in]
% \begin{center}
%     \begin{tikzpicture}[scale=0.5]
%     \draw (0,0) [fill=yellow, color=yellow] rectangle (5,1);
%     \draw (5,0) [fill=yellow, color=yellow] rectangle (10,1);
%     \draw (0,0) rectangle (15,1);
%     \node at (0,0) [anchor=north] {1};
%     \node at (15,0) [anchor=north] {64};
%     \draw (5,0) -- (5,1);
%     \node at (5,0) [anchor=north] {4};
%     \draw (10,0) -- (10,1);
%     \node at (10,0) [anchor=north] {16};
%     \end{tikzpicture}
% \end{center}
% \vfill 


% This is a visual approach to the idea that $\left(\sqrt[3]{64}\right)^2 = 64^{2/3}$   \newline\\

\vspace{0.5in}

In general,
\[
\boxed{\sqrt[\text{root}]{x^{\text{power}}} = x^{\text{power/root}}}
\]
\bigskip 

\begin{example} 
Write each of the following in radical form.	
\begin{multicols}{3}
\begin{enumerate}[(a)]
    \item $5^{1/2}$ 
    \item $(-9)^{5/3}$
    \item $x^{1/3}$
\end{enumerate}	
\end{multicols}
\end{example}
\vfill 

\begin{example} 
Write each of the following using rational exponents.	
\begin{multicols}{3}
\begin{enumerate}[(a)]
    \item $\sqrt{6}$   
    \item $\sqrt[3]{8}$ 
    \item $\sqrt[4]{x^3}$
\end{enumerate}
\end{multicols}
\end{example}
\vfill 

\newpage 



% When dealing with \textbf{even} roots and exponents, keep in mind that the root and exponent don't ``cancel each other out."
% \newline\\

% For instance, $\sqrt{5^2}=\sqrt{25}=5$, but $\sqrt{(-5)^2}=\sqrt{25}=5$.
% \vspace{0.5in}

% The graphs of $y = \sqrt{x^2}$ and $y=|x|$ are shown. Notice they are identical.
% \newline\\

% \begin{tabular}{cc}
%     {\color{blue}$y = \sqrt{x^2}$}    &   {\color{red}$y = |x|$}   \\[11pt]
%     \begin{tikzpicture}[scale=0.65]
%     \begin{axis}
%     [
%         xmin = -5,
%         xmax = 5,
%         ymin = -5,
%         ymax = 5,
%         axis lines = middle,
%         grid,
%         minor tick num = 1
%     ]
%     \addplot [<->, color = blue, line width = 1.5, domain=-4.5:4.5, samples = 300] {abs(x)};
%     \end{axis}
%     \end{tikzpicture}
%     &
%     \begin{tikzpicture}[scale=0.65]
%     \begin{axis}
%     [
%         xmin = -5,
%         xmax = 5,
%         ymin = -5,
%         ymax = 5,
%         axis lines = middle,
%         grid,
%         minor tick num = 1
%     ]
%     \addplot [<->, color = red, line width = 1.5, domain=-4.5:4.5, samples = 300] {abs(x)};
%     \end{axis}
%     \end{tikzpicture}
% \end{tabular}
% \vspace{0.5in}

% Thus, for any real number $x$, $\sqrt{x^2} = |x|$.
% \vspace{0.5in}


% Odd roots, such as $\sqrt[3]{\quad}$ do not follow the same rule as even roots.
% \newline\\

% So for any real number $x$, $\sqrt[3]{x^3} = x$.
% \vspace{1in}

% And, in general, for any real number $x$,	
% \begin{enumerate}
% 	\item If $n$ is even, $\sqrt[n]{x^n} = |x|$.	
% 	\item If $n$ is odd, $\sqrt[n]{x^n} = x$.
% \end{enumerate}
% \vfill

% \emph{Note}: If you are told that variables represent positive values, then you do not need the absolute value bars for even roots. \newline

% \textit{i.e.} If $n$ is even and $x$ is a positive value, then $\sqrt[n]{x^n} = x$.
% \vfill 
% \newpage

\begin{example}
Simplify each of the following. Exact answers only.
\begin{multicols}{2}
\begin{enumerate}[(a)]
    \item $\sqrt{72x^2}$ 
    \item $\sqrt{175x^3}$
\end{enumerate}
\end{multicols}
\vfill 

\begin{multicols}{2}
\begin{enumerate}[(a)]  \setcounter{enumi}{2}
    \item $\sqrt{18x^4}$    
    \item $\sqrt{65x^5y^3}$  
\end{enumerate}
\end{multicols}
\vfill 
\begin{multicols}{2}
\begin{enumerate}[(a)]  \setcounter{enumi}{4}
    \item $\sqrt[3]{27x^7y^8}$
    \item $\sqrt[3]{128x^6}$    
\end{enumerate}
\end{multicols}
\end{example}
\vfill 

% \subsection*{Operations with Radical Expressions.}

% Once we have simplified the radicals, we can add and subtract radical expressions with the same radicands and roots. This is essentially combining like terms.    \vspace{0.75in}


% {\color{red}\textbf{Example 4.}} Simplify each of the following. Exact answers only.	
% \begin{enumerate}[(a)]
%     \item $\sqrt{98} + \sqrt{8}$    \vfill 
%     \item $\sqrt{108x} - \sqrt{300x}$   \vfill 
%     \item $4\sqrt{6} + 3\sqrt{54} - 5\sqrt{45}$ \vfill \newpage
% \end{enumerate}

% We can also multiply and divide radical expressions. \newline\\
% It may be helpful to convert them to rational exponent form first, and then use your laws of exponents.
% \vspace{0.75in}

% {\color{red}\textbf{Example 5.}} Simplify each of the following. Leave no radical expressions in a denominator.	
% \begin{enumerate}[(a)]
%     \item $\sqrt{x^5} \cdot \sqrt{x^2}$ \vfill
%     \item $\sqrt{a^3} \cdot \sqrt[3]{a^2}$  \vfill
%     \item $\sqrt[3]{y^6} \cdot \sqrt{y^3}$  \vfill \newpage
%     \item $\dfrac{\sqrt{x^5}}{\sqrt{x}}$    \vfill
%     \item $\dfrac{7}{\sqrt{2}}$ \vfill
%     \item $\dfrac{\sqrt[3]{y^6}}{\sqrt{y^3}}$  \vfill                 
% \end{enumerate}

\end{document}