\documentclass{article}
\usepackage{amsmath, amssymb, tikzsymbols, bm, tikz, mathtools, tcolorbox, array, sfmath, enumerate, multicol, pgfplots}
\renewcommand{\familydefault}{\sfdefault}
\pgfplotsset{compat=newest}
\usetikzlibrary{arrows.meta}
\everymath{\displaystyle}
\tikzset{>=stealth}
\tikzstyle{input} = [circle, text centered, radius = 1cm, draw = black]
\tikzstyle{function} = [rectangle, text centered, minimum width = 2cm, minimum height = 1cm, draw = black]
\usepackage[top = 0.25in, bottom = 0.25in, left = 1in, right = 1in]{geometry}
\pagestyle{empty}
\raggedright

\newcounter{example}[section]
\newenvironment{example}[1][]{\refstepcounter{example}\par\medskip
   {\color{red}\textbf{Example~\theexample. #1}}}{\medskip}

\begin{document}

\section*{Complex Fractions}

\begin{tcolorbox}[colframe=orange!70!white, coltitle=black, colback = white, title=\textbf{Summary}]
\begin{enumerate}
    \item Complex fractions are fractions within fractions.
    \item Our goal is to have at most 1 fraction bar in our expression.
    \item We can simplify it by methods previously discussed or multiplying every term by the least common tiny denominator (LCTD).
\end{enumerate}
\end{tcolorbox}
\smallskip 

\begin{tcolorbox}[colframe=green!60!black, colback = white,title=\textbf{Complex Fraction}]
A \textbf{complex fraction} is a rational expression which contains other rational expressions in the numerator and/or denominator.
\end{tcolorbox}
\smallskip 


Some examples of complex fractions are given below:

\[
\frac{\frac{2}{x} - 3}{\frac{5}{x} + \frac{7}{x}}
\qquad \text{and} \qquad
\frac{\frac{x}{x+1} + \frac{7}{x}}{\frac{3}{2x} + \frac{8}{x-4}}
\]
\bigskip 

\begin{itemize}
\item Goal is to get {\color{blue}\textbf{at most}} one fraction bar.
\item Clear out ``tiny" fractions by multiplying everything by the {\color{blue}\textbf{least common tiny denominator}}, or {\color{blue}\textbf{LCTD}}.
\item LCTD is the least common denominator of all of the ``tiny" fractions.
\item We then simplify \Cooley
\end{itemize}
\vspace{0.25in}

% Simplifying a complex fraction involves working with the expression until there is \emph{at most} only one fraction bar in the entire expression.   \vspace{0.5in}	

% Recall that when we add or subtract fractions with unlike denominators, we need to find a common denominator first.    \vspace{0.5in}

% We are going to clear out our ``tiny" fractions by multiplying every term by the {\color{blue}\textbf{least common tiny denominator}}, or {\color{blue}\textbf{LCTD}}.
% \vspace{1in} 

% We find the least common tiny denominator by finding the least common denominator of all of the ``tiny" fractions in the expression. We can then simplify, if possible.  	\newline\\	

\subsubsection*{How to find the Least Common (Tiny) Denominators}

\begin{itemize}
\item To find the LCTD of numbers, find the least common multiple of those numbers.  
\item To find the LCTD of variable terms, select the highest power of each term.
\end{itemize}
\vspace{0.25in}

\begin{example}
Simplify each completely. \newline\\
\begin{enumerate}[(a)]
    \item \quad $\dfrac{3+\frac{1}{x}}{\frac{2}{x}+4}$  \newpage
% 	\item \quad $\dfrac{1-\frac{5}{x^2}}{\frac{2}{x^2}-7}$ \vfill 
\begin{multicols}{2}
    \item \quad $\dfrac{\frac{x+4}{3} + \frac{1}{x}}{1 + \frac{1}{x}}$
    \item \quad $\dfrac{\frac{x+4}{5}-\frac{1}{x}}{1-\frac{25}{x^2}}$ 
\end{multicols} 
\vfill 
\begin{multicols}{2}
    \item \quad $\dfrac{\frac{3}{x^2}-\frac{1}{3}}{\frac{x+2}{15}-\frac{1}{x}}$
    \item \quad $\dfrac{\frac{25}{x}-x}{\frac{x-8}{15}+\frac{1}{x}}$
\end{multicols}
    
	% \item \quad $\dfrac{\left(\dfrac{1}{x}+\dfrac{y}{x^2}\right)}{\left(\dfrac{1}{y}+\dfrac{x}{y^2}\right)}$ \vfill 
% 	\item \quad $\dfrac{\left(\dfrac{x}{y}-1\right)}{\left(\dfrac{x^2}{y^2}-1\right)}$ \vfill \newpage 
% 	\item \quad $\dfrac{\left(\dfrac{1}{x+7}-\dfrac{1}{x}\right)}{7}$ 
\end{enumerate}
\end{example}
\vfill 

\end{document}
