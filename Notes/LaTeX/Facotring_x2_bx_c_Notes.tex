\documentclass{article}
\usepackage{amsmath, amssymb, enumerate, tikz, multicol, hyperref, sfmath, pgfplots, tcolorbox}
\pgfplotsset{compat = newest}
\renewcommand{\familydefault}{\sfdefault}
\usepackage[top = 0.5in, bottom=0.5in, right = 1in, left = 1in]{geometry}
\tikzset{>=stealth}
\pagestyle{empty}
\raggedright

\newcounter{example}[section]
\newenvironment{example}[1][]{\refstepcounter{example}\par\medskip
   {\color{red}\textbf{Example~\theexample. #1}}}{\medskip}

\begin{document}

\section*{Factoring \boldmath{$x^2 + bx + c$}}

\begin{tcolorbox}[colframe=orange!70!white, coltitle=black, title=\textbf{Summary}]
\begin{enumerate}
    \item Factoring is ``reverse multiplication."
    \item For $x^2 + bx + c$, find 2 numbers that \textbf{multiply} to make $c$ \underline{and} also \textbf{add} to make $b$.
        \item For numbers, find the greatest common factor (a.k.a. greatest common divisor).
        \item For variables, pick the lowest power of each variable expression.
\end{enumerate}
\end{tcolorbox}
\bigskip 

In the past, you learned about multiplying binomials. \newline\\

\begin{example}
Simplify each.
\begin{multicols}{2}
\begin{enumerate}[(a)]
    \item $(x-3)(x+5)$
    \item $(x-4)(x-7)$
\end{enumerate}
\end{multicols}
\end{example}
\vspace{1in}

Factoring is the \emph{reverse} of the above process. \newline\\

You start with the ``answer" and build the ``question." 
\bigskip 

Your goal is to find 2 numbers that
\begin{itemize}
    \item Multiply to make the last value
    \item Add to make the middle value
\end{itemize}
\bigskip 

\begin{example}
For each of the following find two numbers that meet each requirement.
\begin{multicols}{2}
\begin{enumerate}[(a)]
    \item Multiply to make $-12$; Add to make 1
    \item Multiply to make $-20$; Add to make 8
\end{enumerate} 
\end{multicols}
\vfill 
\begin{multicols}{2}
\begin{enumerate}[(a)]  \setcounter{enumi}{2}
    \item Multiply to make 18; Add to make 7
    \item Multiply to make 45; Add to make 14
\end{enumerate}
\end{multicols}
\vfill
\begin{multicols}{2}
\begin{enumerate}[(a)]  \setcounter{enumi}{4}
    \item Multiply to make 8; Add to make $-9$
    \item Multiply to make $-20$; Add to make $-1$
\end{enumerate}
\end{multicols}
\end{example}
\vfill 
\newpage 

Now we will put the last 2 examples together.
\bigskip 


\begin{example}
Factor each completely.
\begin{multicols}{2}
\begin{enumerate}[(a)]
    \item $x^2+7x-60$    
    \item $x^2+19x-66$   
\end{enumerate}
\end{multicols}
\vfill
\begin{multicols}{2}
\begin{enumerate}[(a)]  \setcounter{enumi}{2}
    \item $x^2-4x-21$  
    \item $x^2+5x+6$
\end{enumerate}
\end{multicols}
\vfill
\begin{multicols}{2}
\begin{enumerate}[(a)]  \setcounter{enumi}{4}
    \item $x^2+6x+8$  
    \item $x^2-14x+24$
\end{enumerate}
\end{multicols}
\end{example}
\vfill

\subsection*{Finding the Greatest Common Factor}

Typically, when the first term is negative, include that negative when factoring out the GCF.
\bigskip 

\begin{example}
Factor the greatest common factor (GCF) from each.
\begin{multicols}{2}
\begin{enumerate}[(a)]
    \item $21x^2 + 28x$
    \item $20x^2+30x$
\end{enumerate}
\end{multicols}
\vfill 
\begin{multicols}{2}
\begin{enumerate}[(a)]  \setcounter{enumi}{2}
    \item $-3x^3+12x^2$
    \item $-2x^3+10x^2$
\end{enumerate}
\end{multicols}
\end{example}

\vfill 
\newpage 

When factoring trinomials, sometimes you can still factor what remains after factoring out the GCF.
\bigskip 

\begin{example}
Factor each \emph{completely}.
\begin{multicols}{2}
\begin{enumerate}[(a)]
    \item $3x^2 + 6x + 48$
    \item $2x^2 - 20x - 120$
\end{enumerate}
\end{multicols}
\vfill 
\begin{multicols}{2}
\begin{enumerate}[(a)]  \setcounter{enumi}{2}
    \item $5x^2 - 45x + 90$
    \item $4x^2 + 12x - 40$
\end{enumerate}
\end{multicols}
\end{example}

\vfill 


% \subsubsection*{Factoring with More Than One Variable}
% \vspace{0.25in}

% If you have to factor something like 
% \[
% x^2 + 15xy + 56y^2
% \]
% \vspace{0.25in}

% Simply
% \begin{enumerate}
%     \item Ignore the other variable 
%     \item Factor like we did   
%     \item Put the other variable back when done
% \end{enumerate}
% \vspace{1.25in}

% \begin{example}
% Factor each completely.
% \begin{multicols}{3}
% \begin{enumerate}[(a)]
%     \item $x^2-5xy+6y^2$    
%     \item $x^2-9xy+20y^2$     
%     \item $a^2 + 6ab - 40b^2$
% \end{enumerate}
% \end{multicols}
% \end{example}

% \vfill 
% \newpage 

% \subsubsection*{Reverse-Engineering to Factor}
% \vspace{0.25in}

% You can also use the graph of the expression to factor to assist you. \newline\\

% You will want to look for where the graph {\color{blue}\textbf{hits the $\mathbf{x}$-axis.}}  \newline\\

% For instance, the graph of $f(x) = x^2 + 2x - 15$ is shown: \newline\\

% \begin{tikzpicture}
% \begin{axis}
% [axis lines = middle, xmin = -6, xmax = 5, xtick distance = 1, grid, legend pos=north west]
% \addplot[<->, thick, blue, domain=-6:5] {x^2+2*x-15};
% \addplot[blue, mark = *, only marks] coordinates {(-5,0) (3,0)};
% \legend{$f(x) = x^2+2x-15$}
% \end{axis}
% \end{tikzpicture}
% \vspace{2in}

\end{document}
