\documentclass{article}
\usepackage{amsmath, amssymb, array, enumerate, tikz, multicol, hyperref, sfmath, pgfplots, tcolorbox}
\pgfplotsset{compat = newest}
\renewcommand{\familydefault}{\sfdefault}
\usepackage[top = 0.5in, bottom=0.5in, right = 1in, left = 1in]{geometry}
\tikzset{>=stealth}
\pagestyle{empty}
\raggedright

\newcounter{example}[section]
\newenvironment{example}[1][]{\refstepcounter{example}\par\medskip
   {\color{red}\textbf{Example~\theexample. #1}}}{\medskip}

\begin{document}

\section*{Solving Quadratic Equations}

\begin{tcolorbox}[colframe=orange!70!white, coltitle=black, title=\textbf{Summary}]
\begin{enumerate}
    \item Quadratic equations are in the form $ax^2 + bx + c = 0$
    \item When solving quadratic equations, \textbf{always} get the equation equal to 0 first.
    % \item For equations in the form $ax^2 + bx + c = 0$, the quadratic formula is 
    % \[
    % x = \frac{-b \pm \sqrt{b^2-4ac}}{2a}, \quad a \neq 0 
    % \]
\end{enumerate}
\end{tcolorbox}

\subsection*{Solving Quadratic Equations by Factoring}

Previously, we looked at factoring quadratic expressions in the forms
\[
x^2 + bx + c \quad \mathrm{and} \quad ax^2 + bx + c
\]
\smallskip 

\parbox{3in}{After getting the equation equal to 0 (if necessary), some equations can quickly be solved by factoring.}
\bigskip 

You will get your equation in the form
\[
({\color{blue}\texttt{expression}})({\color{blue}\texttt{expression}}) = 0
\]

Set each {\color{blue}\texttt{expression}} equal to 0 and solve.
\bigskip 

% \begin{minipage}{0.4\textwidth}
% For $x^2 + bx + c = 0$:
% \begin{itemize}
%     \item The {\color{violet}\textbf{sum}} of the answers is $-b$.   
%     \item The {\color{orange}\textbf{product}} of the answers is $c$
% \end{itemize}
% \end{minipage}
% \hspace{0.1in}
% \begin{minipage}{0.4\textwidth}
% For $ax^2 + bx + c = 0$:
% \begin{itemize}
%     \item The {\color{violet}\textbf{sum}} of the answers is $-\frac{b}{a}$.    
%     \item The {\color{orange}\textbf{product}} of the answers is $\frac{c}{a}$
% \end{itemize}
% \end{minipage}

\begin{example}
Solve each.
\begin{enumerate}[(a)]
\begin{multicols}{3}
    \item $x^2 + x - 12 = 0$
    \item $x^2 + 9x + 10 = 2x$
    \item $7x^2 - 7x - 1 = x^2 + 4$
\end{multicols}
\end{enumerate}
\end{example}
\vfill 

\subsection*{Solving Quadratic Equations That Can't Be Factored}

Many quadratic equations can't be solved by factoring.
\bigskip 

However, after getting the equation equal to 0, \textbf{we can write the other side in vertex form and solve.} \bigskip 

Before that, though, let's take a look at some very basic quadratic equations:

\begin{example}
    Solve each.
\begin{enumerate}[(a)]
\begin{multicols}{3}
    \item $x^2 = 49$
    \item $x^2 = 100$
    \item $(x-1)^2 = 25$
\end{multicols}
\end{enumerate}
\end{example}

\vfill 
\newpage 

\begin{example}
For each of the following,
\begin{enumerate}
    \item Get the equation equal to 0 (if necessary).
    \item Write the other side in vertex form: $a(x-h)^2 + k$
    \item Solve for $x$.
\end{enumerate}
\begin{enumerate}[(a)]
\begin{multicols}{2}
    \item $5x^2 + 3x - 21 = 0$
    \item $8x^2 + 5x - 6 = 0$ 
\end{multicols}
\vfill 
\begin{multicols}{2}
    \item $2x^2 + 9x - 32 = -9$
    \item $-x^2 + 5x - 22 = -8x^2$ 
\end{multicols}
\vfill 
\end{enumerate}
\end{example}


%%%%%

\iffalse

\subsection*{The Quadratic Formula}

For quadratic equations that cannot be factored (and also those that can), there is the {\color{violet}\textbf{quadratic formula}}.
\vspace{1.25in}

\begin{center}
\setlength{\extrarowheight}{5pt}
\begin{tabular}{c|c}
    \textsc{Pros} & \textsc{Cons} \\ \hline 
    Works for {\color{red}\textbf{any}} quadratic equation & Involves some work/calculation \\[5pt]
\end{tabular}
\end{center}
\vspace{2in}

\begin{tcolorbox}[colframe=red!40!blue, coltitle=white, title=\textbf{Quadratic Formula}]
For an equation in the form $ax^2 + bx + c = 0$, where $a \neq 0$,
\[
x = \frac{-b \pm \sqrt{|b|^2-4ac}}{2a}
\]
\end{tcolorbox}

\newpage

\begin{example}
Solve each. Leave square roots in your answers, but simplify what is inside them.
\begin{enumerate}[(a)]
    \item $x^2 - 6x + 3 = 0$    \vfill
    \item $5x^2 + 7x - 5 = 4$   \vfill
    \item $2x^2 - 7x - 7 = 8$   \vfill \newpage
\end{enumerate}
\end{example}

\fi 
%%%%%%%

% \subsection*{The Discriminant}

% In the quadratic formula, 
% \[
% x = \frac{-b \pm \sqrt{b^2-4ac}}{2a}
% \]
% \vspace{0.5in}
% the expression inside the square root
% \[
% b^2 - 4ac
% \]
% \vspace{0.5in}
% is called the {\color{blue}\textbf{discriminant}}.

% \textsc{Properties of the Discriminant}:
% \begin{itemize}
%     \item If $b^2 - 4ac$ is positive, there are 2 different values for $x$.   \vspace{1in}
%     \item If $b^2 - 4ac$ is 0, there is 1 value for $x$ that is repeated.   \vspace{1in}
%     \item If $b^2 - 4ac$ is negative, there are no real values for $x$ (more on this later in the course). \vspace{1in}
%     \item If $\sqrt{b^2-4ac}$ is a positive \emph{rational} number, the original problem can be factored.
% \end{itemize}

% \newpage 

% \subsection*{Solving $\pmb{x^2 + bx + c = 0}$ When $\pmb{a = 1}$}

% $x^2 - 6x + 8 = 0$: \newline\\
% $a = 1, \quad b = -6, \quad c = 8$  

% \vspace{0.75in}

% Recall that 
% \begin{itemize}
%     \item the {\color{violet}\textbf{sum}} of the answers is $-b$.    
%     \item the {\color{orange}\textbf{product}} of the answers is $c$
% \end{itemize}
% \vspace{0.25in}

% \begin{tikzpicture}[scale=0.7]
% \begin{axis}
% [axis lines = middle, xmin = -1, xmax = 5, xtick distance = 1, ymin = -2, ymax = 2]
% \addplot [blue, very thick, samples=200] {x^2-6*x+8};
% \addplot [blue, mark = *, only marks] coordinates {(2,0) (4,0)};
% % \addplot [red, mark = triangle*, mark options = {scale = 1.5, rotate = 180}] coordinates {(3,0)};
% % \node at (axis cs: 3,0) [red, yshift=0.25cm] {$m$};
% \end{axis}
% \end{tikzpicture}

% \vspace{0.5in}

% Let's look at the \emph{mean} of the answers. 

% \begin{align*}
% \text{Mean} &= \text{Sum $\div$ number of values} \\
% \text{Mean} &= -b \div 2 \\
% \text{Mean} &= -\tfrac{b}{2} \\
% \text{Mean} &= \text{$x$-coordinate of vertex}
% \end{align*}

% \vspace{0.25in}

% \begin{tikzpicture}[scale=0.7]
% \begin{axis}
% [axis lines = middle, xmin = -1, xmax = 5, xtick distance = 1, ymin = -2, ymax = 2]
% \addplot [blue, very thick, samples=200] {x^2-6*x+8};
% \addplot [blue, mark = *, only marks] coordinates {(2,0) (4,0)};
% \addplot [red, mark = *, only marks, mark options = {scale = 1.25}] coordinates {(3,-1)} node [below] {$m$};
% \addplot [red, mark = triangle*, mark options = {scale = 1.5, rotate = 180}] coordinates {(3,0)};
% \node at (axis cs: 3,0) [red, yshift=0.25cm] {$m$};
% \end{axis}
% \end{tikzpicture}

% \newpage 

% \begin{center}
% And, in general, for when the distance ($d$) from $m$ to an $x$-intercept is not a whole number:
% \end{center}

% \begin{minipage}{0.3\textwidth}
% \begin{tikzpicture}[scale=0.9]
% \begin{axis}
% [axis lines = middle, xmin = -1, xmax = 5, xtick distance = 1, ymin = -2, ymax = 2]
% \addplot [blue, very thick, samples=200] {x^2-6*x+8};
% \addplot [blue, mark = *, only marks] coordinates {(2,0) (4,0)};
% \addplot [red, mark = *, only marks, mark options = {scale = 1.25}] coordinates {(3,-1)} node [below] {$m$};
% \addplot [red, mark = triangle*, mark options = {scale = 1.5, rotate = 180}] coordinates {(3,0)};
% \node at (axis cs: 3,0) [red, yshift=0.25cm] {$m$};
% \draw [decorate, violet, thick,
%     decoration = {brace, raise=0.45cm}] 
%     (2,0) --  (3,0) 
%     node [pos=0.5, above=15pt, violet] {$\pmb{m-d}$};
% \draw [decorate, violet, thick,
%     decoration = {brace, raise=0.45cm}] 
%     (3,0) --  (4,0) 
%     node [pos=0.5, above=15pt, violet] {$\pmb{m+d}$};
% \end{axis}
% \end{tikzpicture}
% \end{minipage}
% \hspace{0.25in}
% \begin{minipage}{0.4\textwidth}
% \begin{align*}
%     (m+d)(m-d) &= c \\[12pt]
%     % m^2 - d^2 &= c \\[12pt]
%     % d^2 &= m^2 - c \\[12pt]
%     % d &= \sqrt{m^2 - c} \\
% \end{align*}
% \end{minipage}
% % \begin{align*}
% %     x &= m \pm d \\[0.25in]
% %     x &= m \pm \sqrt{m^2-c} \\
% % \end{align*}

% \vspace{1.25in}

% \begin{example}
% Solve each. Exact and simplified answers only.
% \begin{enumerate}[(a)]
% \item $x^2 - 6x + 3 = 0$    \vfill 
% \begin{multicols}{2}
% \item $x^2 + 5x - 1 = 0$   
% \item $x^2 - 9x = -4$
% \end{multicols}
% \end{enumerate} \vfill 
% \end{example}

% \newpage 

% \subsection*{Solving $\pmb{ax^2 + bx + c = 0}$ When $\pmb{a \neq 1}$}

% For those equations where $a \neq 1$,
% \begin{enumerate}
%     \item Divide everything on both sides by $a$
%     \item Proceed as in the previous example
% \end{enumerate}

% \vspace{1in}

% $2x^2 - 5x - 8 = 0$

% \vspace{2.5in}

% \begin{example}
% Solve each. Exact and simplified answers only. \emph{Don't forget to get the equation equal to 0 first.}
% \begin{multicols}{2}
% \begin{enumerate}[(a)] 
%     \item $5x^2 + 7x - 9 = 0$   
%     \item $2x^2 - 7x - 7 = 8$   
% \end{enumerate}
% \end{multicols}
% \vfill 
% \end{example}



\end{document}
