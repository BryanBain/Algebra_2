\documentclass{article}
\usepackage{amsmath, bm, enumerate, multicol, pgfplots, sfmath, tcolorbox}
\pgfplotsset{compat = newest}
\renewcommand{\familydefault}{\sfdefault}
\usepackage[top=0.5in, right=1in, bottom=0.5in, left=1in]{geometry}
\pagestyle{empty}
\raggedright
% \everymath{\displaystyle}

\newcounter{example}[section]
\newenvironment{example}[1][]{\refstepcounter{example}\par\medskip
   {\color{red}\textbf{Example~\theexample. #1}}}{\medskip}

\begin{document}

\section*{Rationalizing Denominators}

\begin{tcolorbox}[colframe=orange!70!white, coltitle=black, title=\textbf{Summary}]
\begin{enumerate}
    \item We multiply the numerator and denominator by a radical expression to rationalize denominators.
    \item The goal is to get rid of any and all radicals in the denominator.
\end{enumerate}
\end{tcolorbox}
\vspace{0.5in}

\parbox{4in}{For the expression $\frac{\sqrt{5}}{\sqrt{2}}$, we can find an equivalent expression that does \\[12pt] not contain a radical in the denominator:}
\vspace{0.5in}

\[
\frac{\sqrt{5}}{\sqrt{2}}
\]
\vspace{0.75in}

The above process is called {\color{blue}\textbf{rationalizing the denominator}}.
\vspace{1in}

\begin{example}
Rationalize each of the following.
\begin{multicols}{2}
\begin{enumerate}[(a)]
    \item $\frac{2}{\sqrt{5}}$
    \item $\frac{2\sqrt{16}}{\sqrt{9x}}$
\end{enumerate}
\end{multicols}
\vfill 
\begin{multicols}{2}
\begin{enumerate}[(a)]  \setcounter{enumi}{2}
    \item $\sqrt{\frac{7x}{3y}}$
    \item $\sqrt[3]{\frac{1}{2}}$
\end{enumerate}
\end{multicols}
\end{example}
\vfill 
\newpage 

How do we go about rationalizing an expression such as $\frac{5}{\sqrt{3}-2}$?
\vspace{4in}

\begin{example}
Rationalize each of the following.
\begin{multicols}{2}
\begin{enumerate}[(a)]
    \item $\frac{2}{3\sqrt{2}+4}$
    \item $\frac{\sqrt{6}+2}{\sqrt{5}-\sqrt{3}}$
\end{enumerate}
\end{multicols}
\end{example}


\end{document}
